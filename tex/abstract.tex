%!TEX root = thesis.tex
\section*{Abstract}

Supernovae are the brightest explosions in the universe. Supernovae in our Galaxy, rare and happening only every few centuries, have probably been observed since the beginnings of mankind. At first they were interpreted as religious omens but in the last half millennium they have increasingly been used to study the cosmos and our place in it. Tycho Brahe deduced from his observations of the famous supernova in 1572, that the stars, in contrast to the widely believe Aristotelian doctrine, were not immutable. 
%This was the first, but not the last time that supernovae had a profound influence on the worldview. After Tycho the geocentric worldview was challenged, by most known by the contemporary Galileo. Galileo perturbed the ancient viewpoint by placing the sun at the centre of the universe. With advances in astronomy, by astronomers like Herschel, it became soon clear that the sun is only one of millions in the Milky Way. In the beginning of last century astronomer Edwin Hubble established that the cosmos did harbour many galaxies and the worldview of a center of the Universe became obsolete. Furthermore the fact that all these other were moving away from the earth suggested a Big Bang and a beginning of the Universe. Soon after Baade and Zwicky established that supernovae are visible in these distant galaxies. Over the next decades these supernovae were studied and  the past few decades that at least one group might help .
More than 400 years after Tycho made his paradigm changing discovery using SN 1572, and some 60 years after supernovae had been identified as distant dying stars, two teams changed the view of the world again using supernovae. The found that the Universe was accelerating in its expansion, a conclusion that could most easily be explained if more than 70\% of the Universe was some previously un-identified form of matter now often referred to as `Dark Energy'.

Beyond their prominent role as tools to gauge our place in the Universe, supernovae themselves have been studied well over the past 75 years. We now know that there are two main physical causes of these cataclysmic events. One of these channels is the collapse of the core of a massive star. The observationally motivated classes Type II, Type Ib and Type Ic have been attributed to these events. This thesis, however is dedicated to the second group of supernovae, the thermonuclear explosions of degenerate carbon and oxygen rich material and lacking hydrogen - called Type Ia supernovae (SNe Ia). 

White dwarf stars are formed at the end of a typical star's life when nuclear burning ceases in the core, the outer envelope is ejected, with the degenerate core typically cooling for eternity. Theory predicts that such stars will self ignite when close to 1.38\,\msun\ (called the Chandrasekhar Mass). Most stars however leave white dwarfs with 0.6\,\msun\, and no star leaves a remnant as heavy as 1.38\msun,  which suggests that they somehow need to acquire mass if they are to explode as SN Ia. Currently there are two major scenarios for this mass acquisition. In the favoured single degenerate scenario the white dwarf accretes matter from a companion star which is much younger in its evolutionary state. The less favoured double degenerate scenario sees the merger of two white dwarfs (with a total combined mass of more than 1.38\,\msun). 

This thesis has tried to answer the question about the mass acquisition in two ways. First the single degenerate scenario predicts a surviving companion post-explosion. We undertook an observational campaign to find this companion in two ancient supernovae (SN 1572 and SN 1006). Secondly, we have extended an existing code to extract the elemental and energy yields of SNe Ia spectra by automating spectra fitting to specific SNe Ia. This type of analysis, in turn, help diagnose to which of the two major progenitor scenarios is right. 

Understanding the progenitors of SN Ia has wide ranging applications. Not only would we better be able to calibrate SNe Ia for use as distance probes, but we could also dramatically improve our understanding of the chemical history of the universe, which SNe Ia play a seminal role in.






%%% Local Variables:
%%% TeX-master: "thesis.tex"
%%% End:
