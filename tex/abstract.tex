%!TEX root = thesis.tex
\section*{Abstract}

Supernovae are the brightest explosions in the universe. Supernovae in our Galaxy, rare and happening only every few centuries, have probably been observed since the beginnings of mankind. At first they were interpreted as religious omens but in the last half millennium they have increasingly been used to study the cosmos and our place in it. Tycho Brahe deduced from his observations of the famous supernova in 1572, that the stars, in contrast to the widely believe Aristotelian doctrine, were not immutable. 
%This was the first, but not the last time that supernovae had a profound influence on the worldview. After Tycho the geocentric worldview was challenged, by most known by the contemporary Galileo. Galileo perturbed the ancient viewpoint by placing the sun at the centre of the universe. With advances in astronomy, by astronomers like Herschel, it became soon clear that the sun is only one of millions in the Milky Way. In the beginning of last century astronomer Edwin Hubble established that the cosmos did harbour many galaxies and the worldview of a center of the Universe became obsolete. Furthermore the fact that all these other were moving away from the earth suggested a Big Bang and a beginning of the Universe. Soon after Baade and Zwicky established that supernovae are visible in these distant galaxies. Over the next decades these supernovae were studied and  the past few decades that at least one group might help .
Nearly 400 years after Tycho made a paradigm changing discovery using SN1572 and more than 60 years after supernovae had been identified as dying stars, two teams - using much more advanced techniques and much more distant supernovae than Tycho - changed the view of the world again. They concluded  that the known matter contributed only five percent to the Universe's make-up. The rest of was consisted of 20\% dark matter and 70\% dark energy.

Despite their prominent role as tools to gauge our place in the Universe, supernovae themselves have been studied over the past 75 years. We now know that there are two main physical causes of these cataclysmic events. The collapse of the core of a massive star is believed to be one of them. The observationally motivated classes Type II, Type Ib and Type Ic have been attributed to this event. This thesis, however is dedicated to the second group of supernovae, the cataclysmic explosion of degenerate carbon and oxygen rich material lacking hydrogen - called Type Ia supernovae (SNe Ia). 

Objects which are rich in oxygen and carbon but lack hydrogen are white dwarfs - the remaining corpses of stars. Theory predicts that they self ignite when close to 1.38\,\msun (called the Chandrasekhar mass). Most stars however leave white dwarfs with 0.6\,\msun\ which suggests that they somehow acquire mass. Currently there are two major scenarios for this mass acquisition. In the favoured single degenerate scenario the white dwarf accretes matter from a companion star which is much younger in its evolutionary state. The less favoured double degenerate scenario sees the merger of two white dwarfs (with a total combined mass of more than 1.38\,\msun). 

This thesis has tried to answer the question about the mass acquisition in two ways. First the single degenerate scenario predicts a surviving companion post-explosion. We undertook an observational campagin to find this companion in two ancient supernovae (SN1572 and SN1006). Secondly, we are modifying an existing code to extract the elemental and energy yields of SNe Ia spectra. This in turn can again indication to which of the two major progenitor scenarios is right. 

The solution to this problem has wide ranging applications. Not only could we calibrate SNe Ia better for use as distance probes, but we could also revise our understanding of the chemical history of the universe, which SNe Ia are an important part of. 






%%% Local Variables:
%%% TeX-master: "thesis.tex"
%%% End:
