% !TEX root =../thesis.tex
\chapter{Introduction}
\label{chap:intro}

\lettrine[lines=4]{F}{or} millenia mankind has watched and studied the nightsky. Apart from planets and comets it appeared an immuatble canvas on which the stars rested. It comes as no surprise that for ancient civilizations supernovae (which were very rare events, occuring only every few centuries ) were interpreted as important Omens as they broke the paradigm of the unchanging nightskies. As these events are so rare their origin remained a mystery until the middle of the last century \citet{1934PNAS...20..254B} suggested that "the phenomenon of a super-nova represents the transition of an ordinary star into a body of considerably smaller mass". For the last 85 years the "supernova-branch" in astronomy has been developing. There have been many advances, but there are still very unknowns about supernovae. This work addresses two subfields of supernovae: The unsolved progenitor problem for Type Ia Supernovae as well as quantifying the nucleosynthetic yield of Type Ia supernovae from low-resolution spectra.


\section{Ancient Supernovae}
\label{sec:ancientsn}

One of the earliest recorded supernovae is SN185. It first appeared in December of 185 and was visible (however fading) till the August of 187. The main record is the \textit{Houhanshu} \citep{2006ChJAA...6..635Z} which had a described it to be close to $\alpha$ \textit{cen}. Follow-up in modern times have revealed a supernova remnant in a distance of roughly 1\;kpc near the $\alpha$ \textit{cen} \citep{2006ChJAA...6..635Z}. SN185 is often named as the oldest written record of a supernova, this is however sometimes contested as it is still not completely clear if the so called "guest star" was a comet or a supernovae. 

The oldest undisputed record of a supernova is SN1006. It was observed worldwide by asian, arabic and european astronomers. 
mention 1885 in andromeda\cite{1885AN....112..360H}


\section{Modern observations and surveys}
\label{sec:}
The most famous and often quoted paper is \citet{1934PNAS...20..254B}. It termed the term supernova and established the difference between common novae and supernovae. \citet{1934PNAS...20..254B} suggested that these luminous events are caused by the death of stars. 

In order to understand the phenomenon of the supernova better Zwicky began a supernova search with the 18-inch Schmidt telescope. He found several supernovae cite?? which in turn inspired Minkowski to classify these supernovae by their spectra \citet{1941PASP...53..224M}. 
He categorized the 14 known objects into two categories. Those without hydrogen he called 'Type I', those with hydrogen he called 'Type II' (see section \ref{????} for a more detailed description).

With the advent of affordable computing in the 1960s the first computer controlled telescopes were build. The 24-inch telecope was built by the Northwestern University and was deployed in Corralitos Observatory in New Mexico. This search resulted 14 supernovae. 

The 1990's can be described as the decade of the supernova surveys. The Leuschner Observatory Supernova Survey began in 1992 followed shortly by the Berkeley Automatic Imaging Telescope (BAIT). These searches resulted in 15 supernovae by 1994 \citep{1994AAS...185.7905V}. One of the most well known discoveries is SN 1994D. This supernovae was observed with the Hubble Space Telescope and resulted in an image that is widely used today (see Figure \ref{fig:sn1994d}).

These successful programmes were succeeded by the Lick Observatory Supernova Search (LOSS) using the Katzman Automatic Imaging Telescope (KAIT). By the year 2000 it had found 96 supernovae \citep{2001ASPC..246..121F}.


\subsection{Novae and Supernovae}
\label{sec:novae_supernovae}

Class aptent taciti sociosqu ad litora torquent per conubia nostra,
per inceptos himenaeos. Donec ante eros, porttitor vel, pellentesque
sit amet, laoreet ut, velit. Vestibulum sit amet turpis sed lorem
vestibulum vulputate. Maecenas sed orci in ante pharetra accumsan. Sed
id nibh. Pellentesque dapibus varius neque. Pellentesque habitant
morbi tristique senectus et netus et malesuada fames ac turpis
egestas. Nullam ultrices augue. Lorem ipsum dolor sit amet,
consectetuer adipiscing elit. Etiam convallis placerat
tortor. Suspendisse potenti. Mauris porttitor, justo et mollis
dapibus, dui nunc accumsan dolor, quis sollicitudin est nisi at
libero. Donec sollicitudin eros sed neque. Nunc at quam.

Donec id arcu. Sed vel sapien sit amet metus vestibulum
fringilla. Etiam fringilla ligula at arcu. Donec bibendum sem et
quam. Nam diam mauris, malesuada vel, placerat a, fermentum sit amet,
lectus. Cras venenatis justo nec leo. Aliquam vulputate erat. Cras
turpis. Cras gravida. Aliquam erat volutpat. Sed porta pretium
ligula. Mauris viverra, nisi euismod vulputate lobortis, est tortor
consectetuer arcu, at congue quam ipsum sit amet sem. Cum sociis
natoque penatibus et magnis dis parturient montes, nascetur ridiculus
mus. Mauris eget dolor.

\section{Supernovae in modern Astrophysics}
\label{sec:sn_modern_astro}
\subsection{Core-Collapse}
nucelosynthesis
first stars
hypernovae
GRB connection
massive stars
expanding photosphere -> distance measurements
binary evoltion

\subsection{Thermonuclear Supernovae}
binary evolution
chandrasekhar
white dwarfs
nucleosynthesis
iron in particular

cosmology


Suspendisse pulvinar. Suspendisse felis nisl, mattis sed, facilisis
at, laoreet vitae, magna. Suspendisse potenti. Pellentesque et ligula
vel mauris suscipit vestibulum. Phasellus eros sem, volutpat at,
feugiat ut, aliquam sed, augue. In hac habitasse platea
dictumst. Suspendisse suscipit. Cum sociis natoque penatibus et magnis
dis parturient montes, nascetur ridiculus mus. In lectus dolor,
commodo non, ultricies eu, scelerisque at, orci. Nulla
semper. Suspendisse potenti. Donec orci diam, pellentesque tristique,
tempus eget, tincidunt in, dolor. Maecenas tristique vehicula
risus. Integer vitae nisi. Aenean sed enim eu nisl suscipit
scelerisque. Integer non metus. Donec dui erat, bibendum eu, suscipit
eu, facilisis non, erat. Morbi dapibus pede id justo. Fusce lobortis
volutpat enim.

Morbi leo turpis, facilisis in, ultrices vel, adipiscing ut,
erat. Praesent ligula. Maecenas quis velit in orci adipiscing
aliquam. Quisque at pede. Integer at odio. Pellentesque feugiat tellus
sed risus. Mauris et turpis. Nam sodales. Suspendisse mollis tincidunt
sapien. In ac sapien et purus sollicitudin ultricies. Integer eget
sapien quis ligula commodo egestas. Nulla aliquam, odio sed tincidunt
blandit, pede dolor gravida nunc, nec condimentum lorem nulla eget
dui.  
\section{Progenitors of Type Ia progenitors}
\label{sec:snia_progenitor}


