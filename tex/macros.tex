% ---------------------------------------------------------------------
% command re-definitions and additions
% ---------------------------------------------------------------------

% i.e. -- e.g. -- etc. -- et. al.
\newcommand{\eg}{{\em e.g.,}}
\newcommand{\ie}{{\em i.e.,}}
\newcommand{\etc}{{\em etc.}}
\newcommand{\etal}{{\em et al.}}

% symbols
\newcommand{\HI}{\hbox{\rmfamily H\,{\textsc i}}}
\newcommand{\HIfat}{\hbox{\rmfamily\bfseries H\,{\textsc i}}}
\newcommand{\HIsub}{\hbox{{\scriptsize H}\,{\tiny I}}}
\newcommand{\HII}{\hbox{\rmfamily H\,{\scshape ii}}}
\newcommand{\HIIsub}{\hbox{\scriptsize \rmfamily H\,{\scshape ii}}}
\newcommand{\Ha}{\hbox{\rmfamily H\,$\alpha$}}
\newcommand{\msun}{\hbox{$M_{\odot}$}}
\newcommand{\mhi}{\hbox{$M_{\HIsub}$}}
\newcommand{\lsun}{\hbox{$L_{\odot}$}}
\newcommand{\mlsun}{(M/L)_{\odot}}
\newcommand{\vexp}{\hbox{$V_{exp}$}}
\newcommand{\vhel}{\hbox{$V_{hel}$}}
\newcommand{\vdisp}{\hbox{$\sigma_{disp}$}}
\newcommand{\Nhi}{\hbox{$N_{\HIsub}$}}
\newcommand{\nhi}{\hbox{$n_{\HIsub}$}}
\newcommand{\Mhi}{\hbox{$M_{\HIsub}$}}
\newcommand{\ra}{$\alpha$}
\newcommand{\dec}{$\delta$}
\newcommand{\degree}{\textdegree}
\newcommand{\arcmin}{\hbox{$^\prime$}}
\newcommand{\arcsec}{\hbox{$^{\prime\prime}$}}
\newcommand{\kms}{\hbox{km s$^{-1}$}}
\newcommand{\mjbeam}{\hbox{mJy beam$^{-1}$}}
\newcommand{\jbeam}{\hbox{Jy beam$^{-1}$}}
\newcommand{\mjbeamkms}{\hbox{mJy/beam km s$^{-1}$}}
\newcommand{\jkms}{\hbox{Jy km s$^{-1}$}}
\newcommand{\coldensity}{\hbox{cm$^{-2}$}}
\newcommand{\voldensity}{\hbox{cm$^{-3}$}}
% add others you need

% footnote symbols
% use \symbolfootnote[1]{footnote} to get an *
%     * 1 - *
%     * 2 - dagger
%     * 3 - double dagger
%     * 4 - ... 9 (see page 175 of the latex manual) 
\long\def\symbolfootnote[#1]#2{\begingroup%
\def\thefootnote{\fnsymbol{footnote}}\footnote[#1]{#2}\endgroup}
% New definition of square root:
% it renames \sqrt as \oldsqrt
% it defines the new \sqrt in terms of the old one
% See:
%  http://en.wikibooks.org/wiki/LaTeX/Tips_and_Tricks#New_Square_Root
\let\oldsqrt\sqrt
\def\sqrt{\mathpalette\DHLhksqrt}
\def\DHLhksqrt#1#2{%
\setbox0=\hbox{$#1\oldsqrt{#2\,}$}\dimen0=\ht0
\advance\dimen0-0.2\ht0
\setbox2=\hbox{\vrule height\ht0 depth -\dimen0}%
{\box0\lower0.4pt\box2}}
% define a new url command so a nice font can be used
\newcommand{\myurl}[1]{\small\texttt{#1}}
% handy referencing of figures/tables, see 'Guide to using Encapsulated
% PostScript in LaTeX.', Section 17.1.1
\newcommand\FigDiff[1]{\hyperref[#1]{Figure~\ref*{#1}} on Page~\pageref*{#1}}
\newcommand\FigSame[1]{\hyperref[#1]{Figure~\ref*{#1}}}
\newcommand\Figref[1]{\ifthenelse{\value{page}=\pageref{#1}}
                     {\FigSame{#1}}{\FigDiff{#1}}}
\newcommand\TabDiff[1]{\hyperref[#1]{Table~\ref*{#1}} on Page~\pageref*{#1}}
\newcommand\TabSame[1]{\hyperref[#1]{Table~\ref*{#1}}}
\newcommand\Tabref[1]{\ifthenelse{\value{page}=\pageref{#1}}
                     {\TabSame{#1}}{\TabDiff{#1}}}

% text to add to end of continued figures caption
\newcommand\ContFig[1]{\textit{(Figure continued from page \pageref*{#1})}}

% ---------------------------------------------------------------------
% pdflatex setup
% ---------------------------------------------------------------------
% make pdflatex use the same spacing (paragraph, line and page breaks)
% as standard LaTeX 
\pdfadjustspacing=1

% ---------------------------------------------------------------------
%  misc. options
% ---------------------------------------------------------------------
% table of contents will go to subsubsections
\setcounter{tocdepth}{3}  

% ---------------------------------------------------------------------
% more liberal 'float' (tables, figures) placement
% ---------------------------------------------------------------------
% Alter some LaTeX defaults for better treatment of figures:
% See p.105 of "TeX Unbound" for suggested values.
% See pp. 199-200 of Lamport's "LaTeX" book for details.
% General parameters, for ALL pages:
\renewcommand{\topfraction}{0.9}	% max fraction of floats at top
\renewcommand{\bottomfraction}{0.8}	% max fraction of floats at bottom
% Parameters for TEXT pages (not float pages):
\setcounter{topnumber}{2}
\setcounter{bottomnumber}{2}
\setcounter{totalnumber}{4}     % 2 may work better
\setcounter{dbltopnumber}{2}    % for 2-column pages
\renewcommand{\dbltopfraction}{0.9}	% fit big float above 2-col. text
\renewcommand{\textfraction}{0.07}	% allow minimal text w. figs
% Parameters for FLOAT pages (not text pages):
\renewcommand{\floatpagefraction}{0.7}	% require fuller float pages
% N.B.: floatpagefraction MUST be less than topfraction !!
\renewcommand{\dblfloatpagefraction}{0.7}	% require fuller float pages
% remember to use [htp] or [htpb] for placement

% ---------------------------------------------------------------------
% Journal name abbreviations
% ---------------------------------------------------------------------
\newcommand{\jnlref}[1]{\textrm{#1}}
\newcommand{\aj}{\jnlref{AJ}}
\newcommand{\araa}{\jnlref{ARA\&A}}
\newcommand{\apj}{\jnlref{ApJ}}
\newcommand{\apjl}{\jnlref{ApJ}}
\newcommand{\apjs}{\jnlref{ApJS}}
\newcommand{\ao}{\jnlref{Appl.~Opt.}}
\newcommand{\apss}{\jnlref{Ap\&SS}}
\newcommand{\aap}{\jnlref{A\&A}}
\newcommand{\aapr}{\jnlref{A\&A~Rev.}}
\newcommand{\aaps}{\jnlref{A\&AS}}
\newcommand{\azh}{\jnlref{AZh}}
\newcommand{\baas}{\jnlref{BAAS}}
\newcommand{\jrasc}{\jnlref{JRASC}}
\newcommand{\memras}{\jnlref{MmRAS}}
\newcommand{\mnras}{\jnlref{MNRAS}}
\newcommand{\pra}{\jnlref{Phys.~Rev.~A}}
\newcommand{\prb}{\jnlref{Phys.~Rev.~B}}
\newcommand{\prc}{\jnlref{Phys.~Rev.~C}}
\newcommand{\prd}{\jnlref{Phys.~Rev.~D}}
\newcommand{\pre}{\jnlref{Phys.~Rev.~E}}
\newcommand{\prl}{\jnlref{Phys.~Rev.~Lett.}}
\newcommand{\pasp}{\jnlref{PASP}}
\newcommand{\pasj}{\jnlref{PASJ}}
\newcommand{\qjras}{\jnlref{QJRAS}}
\newcommand{\skytel}{\jnlref{S\&T}}
\newcommand{\solphys}{\jnlref{Sol.~Phys.}}
\newcommand{\sovast}{\jnlref{Soviet~Ast.}}
\newcommand{\ssr}{\jnlref{Space~Sci.~Rev.}}
\newcommand{\zap}{\jnlref{ZAp}}
\newcommand{\nat}{\jnlref{Nature}}
\newcommand{\iaucirc}{\jnlref{IAU~Circ.}}
\newcommand{\aplett}{\jnlref{Astrophys.~Lett.}}
\newcommand{\apspr}{\jnlref{Astrophys.~Space~Phys.~Res.}}
\newcommand{\bain}{\jnlref{Bull.~Astron.~Inst.~Netherlands}}
\newcommand{\fcp}{\jnlref{Fund.~Cosmic~Phys.}}
\newcommand{\gca}{\jnlref{Geochim.~Cosmochim.~Acta}}
\newcommand{\grl}{\jnlref{Geophys.~Res.~Lett.}}
\newcommand{\jcp}{\jnlref{J.~Chem.~Phys.}}
\newcommand{\jgr}{\jnlref{J.~Geophys.~Res.}}
\newcommand{\jqsrt}{\jnlref{J.~Quant.~Spec.~Radiat.~Transf.}}
\newcommand{\memsai}{\jnlref{Mem.~Soc.~Astron.~Italiana}}
\newcommand{\nphysa}{\jnlref{Nucl.~Phys.~A}}
\newcommand{\physrep}{\jnlref{Phys.~Rep.}}
\newcommand{\physscr}{\jnlref{Phys.~Scr}}
\newcommand{\planss}{\jnlref{Planet.~Space~Sci.}}
\newcommand{\procspie}{\jnlref{Proc.~SPIE}}
\newcommand{\ieeesigprocm}{\jnlref{IEEE~Signal~Processing~Magazine}}
\newcommand{\cjaa}{\jnlref{Chinese J. Astron. Astrophys.}}
\let\astap=\aap
\let\apjlett=\apjl
\let\apjsupp=\apjs
\let\applopt=\ao

%%% Local Variables:
%%% TeX-master: "thesis.tex"
%%% End:
