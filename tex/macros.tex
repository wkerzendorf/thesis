% !TEX root =thesis.tex

% ---------------------------------------------------------------------
% command re-definitions and additions
% ---------------------------------------------------------------------

% RESOLVE IVOA IDENTIFIERS ON THE US-VO DIRECTORY

\newcommand{\ivoa}[1]{\href{http://nvo.stsci.edu/vor10/getRecord.aspx?id=#1}{#1}}

% RESOLVE OBJECTS ON SIMBAD
% Adapted from http://tex.stackexchange.com/questions/4154/how-to-use-newcommand-for-href
% Same functionality as AASTEX macro: http://aastex.aas.org/objects/objectlinking.aas.htm
% #2 is mandatory object name (as written in the text) e.g. \object{IC 2391}
% #1 is optional SIMBAD name (if different from #2) e.g. \object[ETA CHA]{$\eta$~Cha}
\makeatletter
\newcommand{\object}[2][\simbadname]{%
\def\simbadname{#2}%
\StrSubstitute{#1}{ }{+}[\parsedname]%
\href{http://simbad.u-strasbg.fr/simbad/sim-id?Ident=\parsedname}{#2}}%
\makeatother


% i.e. -- e.g. -- etc. -- et. al.
\newcommand{\eg}{{\em e.g.,}}
\newcommand{\ie}{{\em i.e.,}}
\newcommand{\etc}{{\em etc.}}
\newcommand{\etal}{{\em et al.}}

% symbols
\newcommand{\HI}{\hbox{\rmfamily H\,{\textsc i}}}
\newcommand{\HIfat}{\hbox{\rmfamily\bfseries H\,{\textsc i}}}
\newcommand{\HIsub}{\hbox{{\scriptsize H}\,{\tiny I}}}
\newcommand{\HII}{\hbox{\rmfamily H\,{\scshape ii}}}
\newcommand{\HIIsub}{\hbox{\scriptsize \rmfamily H\,{\scshape ii}}}
\newcommand{\Ha}{\hbox{\rmfamily H\,$\alpha$}}
\newcommand{\msun}{\hbox{$M_{\odot}$}}
\newcommand{\mhi}{\hbox{$M_{\HIsub}$}}
\newcommand{\lsun}{\hbox{$L_{\odot}$}}
\newcommand{\mlsun}{(M/L)_{\odot}}
\newcommand{\vexp}{\hbox{$V_{exp}$}}
\newcommand{\vhel}{\hbox{$V_{hel}$}}
\newcommand{\vrot}{\hbox{$v_{\textrm{rot}}$}}
\newcommand{\vrad}{\hbox{$v_{\textrm{rad}}$}}
\newcommand{\vdisp}{\hbox{$\sigma_{disp}$}}
\newcommand{\Nhi}{\hbox{$N_{\HIsub}$}}
\newcommand{\nhi}{\hbox{$n_{\HIsub}$}}
\newcommand{\Mhi}{\hbox{$M_{\HIsub}$}}
\newcommand{\ra}{$\alpha$}
\newcommand{\rasc}[4]{\ensuremath{#1^h#2^m#3\overset{s}{.}#4}}
\newcommand{\decl}[3]{\ensuremath{#1^{\circ}#2\arcmin#3\arcsec}}
\newcommand{\kpc}{kpc}
\newcommand{\dec}{$\delta$}
\newcommand{\degree}{\textdegree}
\newcommand{\arcmin}{\hbox{$^\prime$}}
\newcommand{\arcsec}{\hbox{$^{\prime\prime}$}}
\newcommand{\kms}{\hbox{km s$^{-1}$}}
\newcommand{\mjbeam}{\hbox{mJy beam$^{-1}$}}
\newcommand{\jbeam}{\hbox{Jy beam$^{-1}$}}
\newcommand{\mjbeamkms}{\hbox{mJy/beam km s$^{-1}$}}
\newcommand{\jkms}{\hbox{Jy km s$^{-1}$}}
\newcommand{\coldensity}{\hbox{cm$^{-2}$}}
\newcommand{\voldensity}{\hbox{cm$^{-3}$}}
\newcommand{\moog}{MOOG}
\newcommand{\twomass}{2MASS}
\newcommand{\EM}{electromagnetic}
\newcommand{\xray}{X-ray}
\newcommand{\gammaray}{gamma-ray}
\newcommand{\snratio}{S/N ratio}


% ---- units

%\newcommand{\kms}{ km\,s$^{-1}$}
\newcommand{\masyr}{mas\,yr$^{-1}$}
\newcommand{\myr}{Myr}
\newcommand{\erg}{erg}
%\newcommand{\nir}{NIR}
\newglossaryentry{nir}{name=NIR,description={near infrared},first = {near infrared (NIR)}}
\newglossaryentry{uv}{name=UV,description={ultra violet},first = {ultra violet (UV)}}
\newglossaryentry{halpha}{name=\ensuremath{\textrm{H}_\alpha}, description={First line of the Balmer series at 6563\AA.}}
\newglossaryentry{mgb}{name={Mg \textsc{i} b}, description={Mg I lines  at 5167, 5173, 5184}}
%Mg Ib ??5167, 5173, 5184
\newglossaryentry{sobolevapprox}{name={Sobolev approximation}, description={Sobolev approximation}, first={Sobolev approximation \citep{1960mes..book.....S}}}
\newglossaryentry{radeq}{name=radiative equilibrium, description={radiative equilibrium}}
\newglossaryentry{nebularapprox}{name={nebular approximation}, description={nebular approximation}, first={nebular approximation \citep{1978stat.book.....M}}}
\newglossaryentry{modnebularapprox}{name={modified nebular approximation}, description={modified nebular approximation}, first={modified nebular approximation}}
\newglossaryentry{thompsonscat}{name={Thompson scattering}, description={Thompson scattering}}
\newglossaryentry{lte}{name={LTE}, description={local thermodynamic equilibrium}, first={\textit{local thermodynamic equilibrium} (LTE)}}
\newglossaryentry{mc}{name={MC}, description={Monte Carlo}, first={\textit{Monte Carlo} (MC)}}
\newcommand{\dmb}[1][15]{\ensuremath{\Delta\,m_{#1}(B)}}




%isotopes
\newcommand{\Ni}[1][56]{\ensuremath{^{#1}\textrm{Ni}}}
\newcommand{\Co}[1][56]{\ensuremath{^{#1}\textrm{Co}}}
\newcommand{\Fe}[1][56]{\ensuremath{^{#1}\textrm{Fe}}}
\newcommand{\Na}[1][23]{\ensuremath{^{#1}\textrm{Na}}}
\newcommand{\Ne}[1][20]{\ensuremath{^{#1}\textrm{Ne}}}
\newcommand{\F}[1][19]{\ensuremath{^{#1}\textrm{F}}}
\newcommand{\Ox}[1][16]{\ensuremath{^{#1}\textrm{O}}}
\newcommand{\Carb}[1][12]{\ensuremath{^{#1}\textrm{C}}}
\newcommand{\Mg}[1][24]{\ensuremath{^{#1}\textrm{Mg}}}
\newcommand{\iso}[2]{\ensuremath{^{#2}\textrm{#1}}}
\newcommand{\nififtysix}{$^{56}\textrm{Ni}$}
\newcommand{\cofiftysix}{$^{56}\textrm{Co}$}
\newcommand{\fefiftysix}{$^{56}\textrm{Fe}$}%
 \makeatletter \newcommand{\ion}[2]{#1 \textsc{\@roman{#2}}} \makeatother

% --- telescopes & surveys
\newglossaryentry{scp}{name=SCP,description={Supernova Cosmology Project}, first={Supernova Cosmology Project (SCP)}}
\newglossaryentry{hzsns}{name=HZSNS,description={High Z Supernova Search}, first={High Z Supernova Search (HZSNS)}}
\newglossaryentry{vlt}{name=VLT,description={Very Large Telescope located in Chile}, first={Very Large Telescope (VLT)}}
\newglossaryentry{essence}{name=ESSENCE,description={The Equation of State: SupErNovae trace Cosmic Expansion}, first={`The Equation of State: SupErNovae trace Cosmic Expansion' \citep[ESSENCE;][]{2002AAS...201.7809G}}}
\newglossaryentry{ifu}{name=IFU,description={Integral Field Unit}, first={Integral Field Unit (IFU)}, firstplural={Integral Field Units (IFUs)}} 
\newglossaryentry{hst}{name=HST,description={Hubble Space Telescope}}
\newglossaryentry{snls}{name=SNLS,description={Supernova Legacy Survey\citep{2003AAS...203.8209P}}, first={Supernova Legacy Survey \citep[SNLS;][]{2003AAS...203.8209P}}}
\newglossaryentry{dass}{name=DASS, description={Digitized Astronomy Supernova Survey}, first={Digitized Astronomy Supernova Survey \citep[DASS;][]{1975PASP...87..565C}}}
\newglossaryentry{bait}{name=BAIT, description={Berkley Automatic Imaging Telescope}, first={Berkley Automatic Imaging Telescope \citep[BAIT;][]{1993PASP..105.1164R}}}
\newglossaryentry{kait}{name=KAIT, description={Katzman Automatic Imaging Telescope}, first={Katzman Automatic Imaging Telescope \citep[KAIT;][]{2001ASPC..246..121F}}}
\newglossaryentry{loss}{name=LOSS, description={Lick Observatory Supernova Search}, first={Lick Observatory Supernova Search \citep[LOSS;][]{2000AIPC..522..103L}}}
\newglossaryentry{ctss}{name=CTSS,description={Cal\'{a}n/Tololo supernova survey}, first={Cal\'{a}n/Tololo supernova survey \citep[CTSS;][]{1993AJ....106.2392H}}}
\newglossaryentry{ctio}{name= CTIO, description={Cerro Tololo Inter-American Observatory}, first={Cerro Tololo Inter-American Observatory (CTIO)}}
\newglossaryentry{ptf}{name=PTF, description={Palomar Transient Factory}, first={Palomar Transient Factory \cite[PTF;][]{2009PASP..121.1334R}}}
\newglossaryentry{batse}{name=BATSE, description={Burst and Transient Source Experiment}, first={Burst and Transient Source Experiment (BATSE)}}
\newglossaryentry{bepposax}{name=BeppoSAX, description={named in honor of Giuseppe "Beppo" Occhialini}}
\newglossaryentry{hete2}{name=HETE2, description={High Energy Transient Explorer}, first={High Energy Transient Explorer (HETE)}}
\newglossaryentry{swift}{name=SWIFT, description={Swift }}
\newglossaryentry{evla}{name=EVLA, description={Extended Very Large Array}, first={Extended Very Large Array (EVLA)}}
\newglossaryentry{sdss}{name=SDSS, description={Sloan Digital Sky Survey}}
\newglossaryentry{dss}{name=DSS, description={Digitized Sky Survey}}
\newglossaryentry{skymapper}{name=SkyMapper, description={SkyMapper}, first={SkyMapper \citep{2007PASA...24....1K}}}
\newglossaryentry{panstarrs}{name=PanSTARRS, description={Panoramic Survey Telescope \& Rapid Response System}, first={Panoramic Survey Telescope \& Rapid Response System \citep[PanSTARRS;][]{2004SPIE.5489...11K}}}
\newglossaryentry{lsst}{name=LSST, description={Large Synoptic Survey Telescope}, first={Large Synoptic Survey Telescope \citep[LSST;][]{2006AAS...209.8604P}}}
\newglossaryentry{gaia}{name=GAIA, description={Global Astrometric Interferometer for Astrophysics}, first={Global Astrometric Interferometer for Astrophysics \citep[GAIA;][]{2001A&A...369..339P}}}
\newglossaryentry{ligo}{name=LIGO, description={Laser Interferometer Gravitational Wave Observatory}, first={Laser Interferometer Gravitational Wave Observatory \citep[LIGO;][]{1992Sci...256..325A}}}
\newglossaryentry{aligo}{name=Advanced LIGO, description={Advanced LIGO}, sort=ligo2}
\newglossaryentry{lisa}{name=LISA, description={Laser Interferometer Space Antenna}, first={Laser Interferometer Space Antenna \citep[LISA;][]{1994ESAJ...18..219J}}}
\newglossaryentry{ccd}{name=CCD,description={charged couple device}, first={charged couple device (CCD)}, firstplural={charged couple devices (CCDs)}}
%\newglossaryentry{dass}{name=DASS, description={Digitized Astronomy Supernova Survey}}
%supernova specific
\newcommand{\sn}[2]{\object{SN #1#2}}
\newcommand{\grb}[1]{GRB #1}
%\newglossaryentry{sn}{name=SN, description={supernova}}

% different ias
\newglossaryentry{snia}{name=SN Ia,description={Type Ia supernova},first={Type Ia supernova (henceforth SN Ia)}, firstplural={Type Ia supernovae (henceforth SNe Ia)}, plural=SNe Ia}
\newcommand{\sneia}{\glspl{snia}}
\newcommand{\snia}{\gls{snia}}

\newglossaryentry{branchnormal}{name=\textit{branch-normal}, description=bnormal, first={\textit{branch-normal} SNe Ia \citep{1993AJ....106.2383B}}} 
\newglossaryentry{91t}{name=91T-like, description=91T, first=91T-like} 
\newglossaryentry{91bg}{name=91bg-like, description=91bg, first=91bg-like} 
\newglossaryentry{snib}{name=SN Ib,description={Type Ib supernova},first={Type Ib supernova (henceforth SN Ib)}, firstplural={Type Ib supernovae (henceforth SNe Ib)}, plural=SNe Ib}
\newcommand{\sneib}{\glspl{snib}}
\newcommand{\snib}{\gls{snib}}

\newglossaryentry{snic}{name=SN Ic,description={Type Ic supernova},first={Type Ic supernova (henceforth SN Ic)}, firstplural={Type Ic supernovae (henceforth SNe Ic)}, plural=SNe Ic}
\newcommand{\sneic}{\glspl{snic}}
\newcommand{\snic}{\gls{snic}}

\newglossaryentry{snibc}{name=SN Ib/c,description={Type Ib/c supernova},first={Type Ib/c supernova (henceforth SN Ib/c)}, firstplural={Type Ib/c supernovae (henceforth SNe Ib/c)}, plural=SNe Ib/c}

\newglossaryentry{sniiibc}{name=SN II/Ib/c,description={Type II and Type Ib/c supernova}, plural=SNe II/Ib/c}

\newcommand{\sneibc}{\glspl{snibc}}
\newcommand{\snibc}{\gls{snibc}}


\newglossaryentry{snii}{name=SN II,description={Type II supernova},first={Type II supernova (henceforth SN II)}, firstplural={Type II supernovae (henceforth SNe II)}, plural=SNe II}
\newcommand{\sneii}{\glspl{snii}}
\newcommand{\snii}{\gls{snii}}

\newglossaryentry{sniib}{name=SN IIb,description={Type IIb supernova},first={Type IIb supernova (SN IIb)}, firstplural={Type IIb supernovae (SNe IIb)}, plural=SNe IIb}
\newcommand{\sneiib}{\glspl{sniib}}
\newcommand{\sniib}{\gls{sniib}}

%iip
\newglossaryentry{sniip}{name=SN IIP,description={Type IIP supernova},first={Type IIP supernova (henceforth SN IIP)}, firstplural={Type II Plateau supernovae \citep[SNe IIP;][]{1979A&A....72..287B}}, plural=SNe IIP}
\newcommand{\sneiip}{\glspl{sniip}}
\newcommand{\sniip}{\gls{sniip}}

%iil
\newglossaryentry{sniil}{name=SN IIL,description={Type IIL supernova},first={Type IIL supernova (henceforth SN IIL)}, firstplural={Type II Linear supernovae \citep[SNe IIL;][]{1990MNRAS.244..269S})}, plural=SNe IIL}
\newcommand{\sneiil}{\glspl{sniil}}
\newcommand{\sniil}{\gls{sniil}}

%iin
\newglossaryentry{sniin}{name=SN IIn,description={Type IIn supernova},first={Type II narrow-lined supernova (SN IIn)}, firstplural={Type IIn supernovae (henceforth SNe IIn)}, plural=SNe IIn}
\newcommand{\sneiin}{\glspl{sniin}}
\newcommand{\sniin}{\gls{sniin}}



%

\newglossaryentry{dtd}{name=DTD,description={delay time distribution},first={delay time distribution (henceforth DTD)}, firstplural={delay time distributions (henceforth DTDs)}, plural=DTDs}
\newcommand{\dtd}{\gls{dtd}}

\newglossaryentry{hvg}{name=HVG,description={high velocity group},first={high velocity group (henceforth HVG)}, firstplural={high velocity groups (henceforth HVGs)}, plural=HVGs}
\newcommand{\hvg}{\gls{hvg}}

\newglossaryentry{lvg}{name=LVG,description={low velocity group},first={low velocity group (henceforth LVG)}, firstplural={low velocity groups (henceforth LVGs)}, plural=LVGs}
\newcommand{\lvg}{\gls{lvg}}


\newglossaryentry{snr}{name=SNR,description={supernova remnant},first={supernova remnant (henceforth SNR)}, firstplural={supernova remnants (henceforth SNRs)}}
\newcommand{\snr}{\gls{snr}}

%white dwarfs

\newglossaryentry{onemgwd}{name=ONe-WD,description={Oxygen/Neon White Dwarf}}

\newglossaryentry{cowd}{name=CO-WD, description={Carbon/Oxygen White Dwarf}}


%\newglossaryentry{wd}{name=WD,description={White Dwarf}}


\newglossaryentry{sds}{name=SD-Scenario,description={single degenerate scenario (single white dwarf accreting from non-degenerate companion)}, first={single degenerate scenario (SD-scenario)}}
\newglossaryentry{sss}{name=SSS,description={supersoft X-ray source}, firstplural={supersoft X-ray sources (SSS)}}

\newglossaryentry{dds}{name=DD-Scenario, description={double degenerate scenario (merging of two white dwarfs)}, first={double degenerate scenario (DD-scenario)}}


\newglossaryentry{amcvn}{name=AM CVn, description={AM Canum Venaticorum star (white dwarf accreting hydrogen poor matter from a companion star; see \cite{2005ASPC..330...27N})}}


\newglossaryentry{rlof}{name=RLOF,description={Roche Lobe Overflow (see \citet{1971ARA&A...9..183P} for a more detailed description)}, first={Roche Lobe Overflow (RLOF)}}

\newglossaryentry{mchan}{name=$M_\textrm{Chan}$,description={Chandrasekhar mass ($1.38\,M_\odot$; see \citet{1931ApJ....74...81C})}, first={Chandrasekhar mass \citep[$M_\textrm{Chan}=1.38\,M_\odot$;][]{1931ApJ....74...81C}} sort=mchan}

\newglossaryentry{w7}{name=W7 model,description={W7},first = {W7 model \citep{1984ApJ...286..644N}}}



%\newcommand{\onemgwd}{ONe-WD}
%\newcommand{\cowd}{CO-WD}
%\newcommand{\WD}{WD}
\newcommand{\sd}{SD}
\newcommand{\dd}{DD}
%\newcommand{\amcvn}{AM CVn}
%\newcommand{\sss}{SSS}
%\newcommand{\rlof}{RLOF}
%\newcommand{\mchan}{$M_\textrm{Chan}$}

%\newglossaryentry{scp}{name=SCP,description={Supernova Cosmology Project}}

%SN1572 specific 
\newcommand{\stara}{Tycho-A}
\newcommand{\staraii}{Tycho-A2}
\newcommand{\starb}{Tycho-B}
\newcommand{\starc}{Tycho-C}
\newcommand{\stard}{Tycho-D}
\newcommand{\stare}{Tycho-E}
\newcommand{\starg}{Tycho-G}

%\def\m15{\ensuremath{\Delta\,M_{15}}}
%\newcommand{\mym}[1][15]{\ensuremath{\Delta\,M_{#1}}}
%SN1006 specific

\newcommand{\smstar}{\object[NAME SM STAR]{SM-Star}}
\newcommand{\candstar}[1]{SN1006-#1}


%common papers

\newcommand{\rl}{RP04}
\newcommand{\gh}{GH09}
\newcommand{\wek}{WEK09}

%citepapers
\newcommand{\citescipy}{\citep{Jones:2001fk}}
\newcommand{\citemoog}{\citep{1973ApJ...184..839S}}
\newcommand{\citesfit}{\citep{2001A&A...376..497J}}

%general astrophysics
\newglossaryentry{agb}{name=AGB,description={asymptotic giant branch}}
\newglossaryentry{csm}{name=CSM,description={circumstellar medium}, first={circumstellar medium (CSM)}}
\newglossaryentry{csi}{name=CSI,description={circumstellar interaction}, first={circumstellar interaction (CSI)}}
\newglossaryentry{ism}{name=ISM,description={interstellar medium}, first={interstellar medium (ISM)}}
\newglossaryentry{ige}{name=IGE,description={iron group element}, first={iron group element (IGE)}, firstplural={iron group elements (IGEs)}}
\newglossaryentry{epm}{name=EPM,description={Expanding Photosphere Method}, first={Expanding Photosphere Method (EPM)}}
\newglossaryentry{aic}{name=AIC,description={Accretion Induced Collapse}, first={accretion induced collapse (AIC)}}
\newglossaryentry{ime}{name=IME,description={intermediate mass element}, first={intermediate mass element (IME)}, firstplural={intermediate mass elements (IMEs)}}
\newglossaryentry{h0}{name=\ensuremath{H_0},description={Hubbles constant}}
\newglossaryentry{nse}{name=NSE,description={nuclear statistical equilibrium}, first={nuclear statistical equilibrium (NSE)}}
\newglossaryentry{cdm}{name=CDM,description={Cold Dark Matter}}
\newglossaryentry{grb}{name=GRB,description={Gamma Ray Burst}, first={Gamma Ray Burst (GRB)}, firstplural={Gamma Ray Bursts (GRBs)}}
\newglossaryentry{mlcs}{name=mlcs,description={Multicolor Light-Curve Shape}, first={Multicolor Light-Curve Shape method \citep[MLCS;][]{1996ApJ...473...88R}}}
\newglossaryentry{rsoph}{name=RS Oph ,description={RS??}}
\newglossaryentry{usco}{name=U Sco,description={U??}}
\newglossaryentry{casa}{name=Cas A,description={Cassiopeia A}}
\newglossaryentry{cepheid}{name=Cepheid,description={Cepheid}}
\newglossaryentry{urca}{name=\textit{Urca}, description={process predominatly contributing to cooling. The \textit{Urca} process consists of alternating electron-capture and $\beta^{-}$ decay of two nuclei pairs.},sort=urca} 
\newglossaryentry{alphacen}{name=Alpha Centauri,description={Alpha Centauri}}
\newglossaryentry{pcygni}{name=P Cygni,description={a hypergiant luminous blue variable with strong winds. often referred to as a description for line profiles}}
\newglossaryentry{teff}{name=\ensuremath{T_\textrm{eff}}, description={effective temperature - Temperature of a blackbody emitting the same total energy.}, sort=teff}
\newglossaryentry{logg}{name=\ensuremath{\log{g}},description={surface gravity}}
\newglossaryentry{feh}{name=[Fe/H],description={iron abundance relative to the sun}, sort=feh}
\newglossaryentry{texp}{name=\ensuremath{t_{\rm exp}},description={time since explosion (measured in days)}}
\newglossaryentry{lmc}{name=LMC,description={Large Magellanic Cloud}, first={Large Magellanic Cloud (LMC)}}
\newglossaryentry{smc}{name=SMC,description={Small Magellanic Cloud}}
\newglossaryentry{wd}{name=WD, description={extremely dense stellar remnant}, first={white dwarf (WD)}}
\newglossaryentry{z}{name=\ensuremath{z},description={redshift}, sort=z}
%\newcommand{\lmc}{LMC}
%\newcommand{\smc}{SMC}


%\newcommand{\agb}{AGB}
%\newcommand{\csm}{CSM}
%\newcommand{\csi}{CSI}
%\newcommand{\ism}{ISM}
%\newcommand{\ige}{IGE}
%\newcommand{\nse}{NSE}
%\newcommand{\ime}{IME}

%\newcommand{\grb}{GRB}
%\newcommand{\pcygni}{P Cygni}
%\newcommand{\urca}{\gls{urca}}
%\newcommand{\teff}{\ensuremath{T_{\rm eff}}}
%\newcommand{\logg}{\ensuremath{\log{g}}}
%\newcommand{\feh}{[Fe/H]}
\newcommand{\lum}{\ensuremath{L}}
%\newcommand{\texp}{\ensuremath{t_{\rm exp}}}
\newcommand{\loglbol}{\ensuremath{\log{L/L_\odot}}}

%dalek chapter
\newglossaryentry{ga}{name=GA,description={genetic algorithm}}

\newglossaryentry{individual}{name=individual,description={single representation of a parameter set}}
\newglossaryentry{gene}{name=gene,description={Genetic Algorithm term for an individual parameter}}
\newglossaryentry{genome}{name=genome,description={describes the parameter set of an indivdual}}
\newglossaryentry{fitness}{name=fitness,description={goodness of fit term in Genetic Algorithms}}
\newglossaryentry{genotype}{name=genotype,description={vectorized representation of the parameters}}
\newglossaryentry{phenotype}{name=phenotype,description={solution to a set of parameters}}
\newglossaryentry{crossover}{name=crossover,description={binary operator to create new individual from two or more old individuals}}
\newglossaryentry{mutation}{name=mutation,description={unary operator to alter a newly created individual}}
\newglossaryentry{child}{name=child, plural=children, description={member of the new generation created by parents (members of the old generation)}}
\newglossaryentry{parent}{name=parent, description={member of the old generation creating a child (member of the next generation)}}
\newglossaryentry{api}{name=API, description={application program interface}, first={application program interface (API)}}
\newglossaryentry{nfl}{name=`No Free Lunch', description={`No Free Lunch' theorem described in \citet{DBLP:journals/tec/DolpertM97}}, first={`No Free Lunch' theorem \citep{DBLP:journals/tec/DolpertM97}}, sort=nofreelunch}
\newglossaryentry{superindividual}{name=\textit{superindividual},description={Individual with extremely high fitness, which has only one parameter close to optimal}, sort=superindividual}
\newglossaryentry{elitism}{name=elitism,description={elitism}}
\newglossaryentry{rws}{name=RWS,description={roulette wheel selection}, first={roulette wheel selection (RWS)}}
\newglossaryentry{tsp}{name=TSP,description={Travelling Salesman Problem}}
%\newglossaryentry{synow}{name=SYNOW,description={Small Magellanic Cloud}}
\newglossaryentry{dalek}{name=\textsc{Dalek} code,description={Automatic SN Ia spectrum fitting code}, sort=dalek}
\newglossaryentry{mlc}{name=MLMC,description={Mazzali and Lucy SN Ia spectrum synthesis code} , first={`Mazzali and Lucy Monte Carlo' code (MLMC)}}
\newglossaryentry{vph}{name={\ensuremath{v_\textrm{ph}}}, description={photospheric velocity of a supernova}, sort=vph}
\newglossaryentry{lbol}{name={\ensuremath{L_\textrm{bol}}}, description={bolometric luminosity}, sort=lbol}

%\newcommand{\ga}{GA}
%\newcommand{\tsp}{TSP}
%\newcommand{\dalek}{Dalek Code}
\newcommand{\synow}{SYNOW}
%\newcommand{\vph}{\ensuremath{v_{\rm ph}}}
%\newcommand{\lbol}{\ensuremath{L_{\rm bol}}}


%maths
\newcommand{\deltri}{Delauney Triangulation}
% footnote symbols
% use \symbolfootnote[1]{footnote} to get an *
%     * 1 - *
%     * 2 - dagger
%     * 3 - double dagger
%     * 4 - ... 9 (see page 175 of the latex manual) 
\long\def\symbolfootnote[#1]#2{\begingroup%
\def\thefootnote{\fnsymbol{footnote}}\footnote[#1]{#2}\endgroup}
% New definition of square root:
% it renames \sqrt as \oldsqrt
% it defines the new \sqrt in terms of the old one
% See:
%  http://en.wikibooks.org/wiki/LaTeX/Tips_and_Tricks#New_Square_Root
\let\oldsqrt\sqrt
\def\sqrt{\mathpalette\DHLhksqrt}
\def\DHLhksqrt#1#2{%
\setbox0=\hbox{$#1\oldsqrt{#2\,}$}\dimen0=\ht0
\advance\dimen0-0.2\ht0
\setbox2=\hbox{\vrule height\ht0 depth -\dimen0}%
{\box0\lower0.4pt\box2}}
% define a new url command so a nice font can be used
\newcommand{\myurl}[1]{\small\texttt{#1}}
% handy referencing of figures/tables, see 'Guide to using Encapsulated
% PostScript in LaTeX.', Section 17.1.1
\newcommand\FigDiff[1]{\hyperref[#1]{Figure~\ref*{#1}} on Page~\pageref*{#1}}
\newcommand\FigSame[1]{\hyperref[#1]{Figure~\ref*{#1}}}
\newcommand\Figref[1]{\ifthenelse{\value{page}=\pageref{#1}}
                     {\FigSame{#1}}{\FigDiff{#1}}}
\newcommand\TabDiff[1]{\hyperref[#1]{Table~\ref*{#1}} on Page~\pageref*{#1}}
\newcommand\TabSame[1]{\hyperref[#1]{Table~\ref*{#1}}}
\newcommand\Tabref[1]{\ifthenelse{\value{page}=\pageref{#1}}
                     {\TabSame{#1}}{\TabDiff{#1}}}

% text to add to end of continued figures caption
\newcommand\ContFig[1]{\textit{(Figure continued from page \pageref*{#1})}}

% ---------------------------------------------------------------------
% pdflatex setup
% ---------------------------------------------------------------------
% make pdflatex use the same spacing (paragraph, line and page breaks)
% as standard LaTeX 
\pdfadjustspacing=1

% ---------------------------------------------------------------------
%  misc. options
% ---------------------------------------------------------------------
% table of contents will go to subsubsections
\setcounter{tocdepth}{3}  

% ---------------------------------------------------------------------
% more liberal 'float' (tables, figures) placement
% ---------------------------------------------------------------------
% Alter some LaTeX defaults for better treatment of figures:
% See p.105 of "TeX Unbound" for suggested values.
% See pp. 199-200 of Lamport's "LaTeX" book for details.
% General parameters, for ALL pages:
\renewcommand{\topfraction}{0.9}	% max fraction of floats at top
\renewcommand{\bottomfraction}{0.8}	% max fraction of floats at bottom
% Parameters for TEXT pages (not float pages):
\setcounter{topnumber}{2}
\setcounter{bottomnumber}{2}
\setcounter{totalnumber}{4}     % 2 may work better
\setcounter{dbltopnumber}{2}    % for 2-column pages
\renewcommand{\dbltopfraction}{0.9}	% fit big float above 2-col. text
\renewcommand{\textfraction}{0.07}	% allow minimal text w. figs
% Parameters for FLOAT pages (not text pages):
\renewcommand{\floatpagefraction}{0.7}	% require fuller float pages
% N.B.: floatpagefraction MUST be less than topfraction !!
\renewcommand{\dblfloatpagefraction}{0.7}	% require fuller float pages
% remember to use [htp] or [htpb] for placement

% ---------------------------------------------------------------------
% Journal name abbreviations
% ---------------------------------------------------------------------
\newcommand{\jnlref}[1]{\textrm{#1}}
\newcommand{\aj}{\jnlref{AJ}}
\newcommand{\araa}{\jnlref{ARA\&A}}
\newcommand{\apj}{\jnlref{ApJ}}
\newcommand{\apjl}{\jnlref{ApJ}}
\newcommand{\apjs}{\jnlref{ApJS}}
\newcommand{\ao}{\jnlref{Appl.~Opt.}}
\newcommand{\apss}{\jnlref{Ap\&SS}}
\newcommand{\aap}{\jnlref{A\&A}}
\newcommand{\aapr}{\jnlref{A\&A~Rev.}}
\newcommand{\aaps}{\jnlref{A\&AS}}
\newcommand{\azh}{\jnlref{AZh}}
\newcommand{\baas}{\jnlref{BAAS}}
\newcommand{\jrasc}{\jnlref{JRASC}}
\newcommand{\memras}{\jnlref{MmRAS}}
\newcommand{\mnras}{\jnlref{MNRAS}}
\newcommand{\pra}{\jnlref{Phys.~Rev.~A}}
\newcommand{\prb}{\jnlref{Phys.~Rev.~B}}
\newcommand{\prc}{\jnlref{Phys.~Rev.~C}}
\newcommand{\prd}{\jnlref{Phys.~Rev.~D}}
\newcommand{\pre}{\jnlref{Phys.~Rev.~E}}
\newcommand{\prl}{\jnlref{Phys.~Rev.~Lett.}}
\newcommand{\pasp}{\jnlref{PASP}}
\newcommand{\pasj}{\jnlref{PASJ}}
\newcommand{\qjras}{\jnlref{QJRAS}}
\newcommand{\skytel}{\jnlref{S\&T}}
\newcommand{\solphys}{\jnlref{Sol.~Phys.}}
\newcommand{\sovast}{\jnlref{Soviet~Ast.}}
\newcommand{\ssr}{\jnlref{Space~Sci.~Rev.}}
\newcommand{\zap}{\jnlref{ZAp}}
\newcommand{\nat}{\jnlref{Nature}}
\newcommand{\iaucirc}{\jnlref{IAU~Circ.}}
\newcommand{\aplett}{\jnlref{Astrophys.~Lett.}}
\newcommand{\apspr}{\jnlref{Astrophys.~Space~Phys.~Res.}}
\newcommand{\bain}{\jnlref{Bull.~Astron.~Inst.~Netherlands}}
\newcommand{\fcp}{\jnlref{Fund.~Cosmic~Phys.}}
\newcommand{\gca}{\jnlref{Geochim.~Cosmochim.~Acta}}
\newcommand{\grl}{\jnlref{Geophys.~Res.~Lett.}}
\newcommand{\jcp}{\jnlref{J.~Chem.~Phys.}}
\newcommand{\jgr}{\jnlref{J.~Geophys.~Res.}}
\newcommand{\jqsrt}{\jnlref{J.~Quant.~Spec.~Radiat.~Transf.}}
\newcommand{\memsai}{\jnlref{Mem.~Soc.~Astron.~Italiana}}
\newcommand{\nphysa}{\jnlref{Nucl.~Phys.~A}}
\newcommand{\physrep}{\jnlref{Phys.~Rep.}}
\newcommand{\physscr}{\jnlref{Phys.~Scr}}
\newcommand{\planss}{\jnlref{Planet.~Space~Sci.}}
\newcommand{\procspie}{\jnlref{Proc.~SPIE}}
\newcommand{\ieeesigprocm}{\jnlref{IEEE~Signal~Processing~Magazine}}
\newcommand{\cjaa}{\jnlref{Chinese J. Astron. Astrophys.}}
\newcommand{\pasa}{\jnlref{PASA}}
\let\astap=\aap
\let\apjlett=\apjl
\let\apjsupp=\apjs
\let\applopt=\ao

%%% Local Variables:
%%% TeX-master: "thesis.tex"
%%% End: