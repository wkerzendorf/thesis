%!TEX root = single_chapter_conclusion.tex
\chapter{Conclusion and Future Work}
\label{chap:four}

A lot of the aspects of \snia physics have been solved. We are very certain that this phenomenon is powered by the nuclear burning of degenerate Carbon/Oxygen fuel. The lack of hydrogen in the spectra of \sneia suggests that the progenitors are white dwarfs as they do not posses an extensive hydrogen envelope. The question that remains is: How and why do these objects explode?

\section{Single or Double Degenerate?}

The initial idea for this thesis was simple: High resolution spectroscopy and high precision astrometry of close and young remnant should reveal the suggested donor star. At this time the only viable scenario was the \sd-scenario. \dd-scenarios were almost unanimously believed to lead to accretion induced collapse and not a \snia. The first project was to confirm \starg\ as the progenitor of SN1572 and then move on and find the progenitors in both SN1006 and SN1604. 

The observations however started to show a completely different picture. The very unusual kinematics claimed for \starg \citep{2004Natur.431.1069R}, was only slightly unusual. The Besan\c{c}on model suggested the unusual velocity to be very usual if \starg\ was an uninvolved background interloper (see Section \ref{sec:sn1006:interloper}). We have realized that kinematics is definitley not conclusive evidence for a progenitor hunt, at most it is suggestive. 

The \sd-scenario in most cases suggests \rlof as the mass transfer. A consequence of this mass transfer mode is tidal coupling of the donor. Tidal coupling also implies that the rotation of the donor post-explosion is coupled to its escape velocity (see Section \ref{sec:sn1572_starg_rotation} and Figure \ref{fig:theorot}). For our search a serendipitous coincidence is that most low-mass stars do not rotate. Our work in Chapter \ref{chap:sn1572_starg} suggested that \starg does not have a unusually high rotation. Further work with better data (see Chapter \ref{chap:sn1572_hires}) established this. It is not entirely without caveats: the rotation might vanish post-explosion when the star puffs-up. In addition, the observable is not the rotation but the projected rotation, which is the intrinsic rotation diminished by a factor $\sin{i}$. Ironically, after the establishment of rotation as a donor star feature, we discovered a star with an unusually high rotation which was kinematically very normal (see Chapter \ref{chap:sn1572_hires}). No other stars in SN1572 and SN1006 (from preliminary analysis) show any unusually high rotation. 

Finding donor stars seems much harder than we initially thought. Both SN1572 and SN1006 don't have any strong candidates. We do acknowledge unusual stars in these remnants, but all of them have believable alibis which do not involve a \snia explosion. Only one star in the well studied remnant of SN1572 has so far eluded our spectroscopic scrutiny. This star (\stara 2 by our nomenclature) has an unusual proper motion according to our HST astrometry (see \ref{fig:propmot_sn1572_hires}). In addition it is located very close to the \xray center of SN1572. Unfortunately it hides 0.4\arcsec away from the 4 magnitudes brighter \stara (see Figure \ref{fig:stara2_overview}). This makes it impossible to obtain ground-based optical spectroscopy, but offers the possibility for challenging infrared observations aided by adaptive-optics. We are currently running a GNIRS-based campaign to obtain the fundamental stellar parameters, radial velocity and rotation. 

The bottom line for both SN1572 and SN1006 seems to be: No bright progenitors. There are as always caveats, but we do think that with current theoretical knowledge and current instrumentation it is not worthwhile to scrutinize these remnants further. The only exception is to undertake photometrically deep observations of these remnants and look for a hot white dwarf. 

\section{The curious case of Kepler}

The last of the young remnants and the most distant one \citep[][estimates a distance of $\ge 6$\,\kpc]{2008ApJ...689..231V}.  The morphology of this remnant is not as clean and spherical as for SN1006 and SN1572 (see Figure \ref{fig:sn1604_observations}). For a long time SN1604 (Kepler's supernova) was also believed to be a \snib, but prominent iron emission in \xray spectra \citep{2007ApJ...668L.135R} suggests this event to be a \snia \citep{1995ApJ...444L..81H}. SN1604 also shows an abundance in nitrogen which is unusual for a \snia. \citet{1991ApJ...366..484B} and \citet{2003A&A...407..249S} suggest that the remnant itself posseses a very high systemic velocity of  $\approx 250\,\kms$. A recent study by \citet{2011arXiv1103.5487C} suggests that a \sd-scenario with an AGB star as a donor would explain all the observed peculiarities. They make the prediction that this star should be visible and very bright post-explosion. 

We have obtained data with the WiFeS integral field spectrograph. The field of view for this instrument. is rectangular with dimensions of $25\arcsec \times 38\arcsec$. We have two overlapping fields that cover all stars at a projected velocity of 1300\,\kms (assuming a distance of 6\,\kpc). Extracting stellar spectra from our poor seeing data ($> 1.5\arcsec$) is technically challenging and we have not invested t


\section{Divide and impera}

\section{Dalek}





grand ideas about progenitor mass
